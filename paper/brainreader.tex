\documentclass[journal]{IEEEtran}

\usepackage{cite}
\usepackage[pdftex]{graphicx}
\graphicspath{{figures/}}
\DeclareGraphicsExtensions{.pdf,.jpeg,.jpg,.png}
\usepackage[cmex10]{amsmath}
\usepackage{algorithmic}
\usepackage[caption=false,font=normalsize,labelfont=sf,textfont=sf]{subfig}
\usepackage{fixltx2e}
%\usepackage{stfloats}


% correct bad hyphenation here
\hyphenation{op-tical net-works semi-conduc-tor}

% Load basic packages
\usepackage{balance}  % to better equalize the last page
\usepackage{graphics} % for EPS, load graphicx instead
\usepackage{times}    % comment if you want LaTeX's default font
\usepackage{url}      % llt: nicely formatted URLs
%\usepackage{flushend} % bjoern: attempt to balance last page
\usepackage[]{algorithm2e}
\let\proof\relax
\let\endproof\relax
\usepackage{amsthm}
\newtheorem{theorem}{Theorem}

% llt: Define a global style for URLs, rather that the default one
\makeatletter
\def\url@leostyle{%
  \@ifundefined{selectfont}{\def\UrlFont{\sf}}{\def\UrlFont{\small\bf\ttfamily}}}
\makeatother
\urlstyle{leo}

\usepackage[pdftex]{hyperref}
\hypersetup{
pdftitle={SIGCHI Conference Proceedings Format},
pdfauthor={LaTeX},
pdfkeywords={SIGCHI, proceedings, archival format},
bookmarksnumbered,
pdfstartview={FitH},
colorlinks,
citecolor=black,
filecolor=black,
linkcolor=black,
urlcolor=black,
breaklinks=true,
}

\begin{document}

%!TEX root = brainreader.tex

%use these commands while writing
\newcommand {\valkyrie}[1]{{\bf{VS: #1}\normalfont}}
\newcommand {\natalia}[1]{{\bf{NB: #1}\normalfont}}
%\newcommand {\changes}[1]{{\color{changes}{#1}\normalfont}}
\newcommand {\changes}[1]{{#1}}

%uncomment these for final submit
%\renewcommand {\valkyrie}[1]{}
%\renewcommand {\natalia}[1]{}

\newcommand{\noopsort}[2]{#2}

\newcommand {\bt}[1]{\textbf{#1} \normalfont}
\newcommand{\squishlist}{
 \begin{list}{$\bullet$}
  { \setlength{\itemsep}{0pt}
     \setlength{\parsep}{3pt}
     \setlength{\topsep}{3pt}
     \setlength{\partopsep}{0pt}
     \setlength{\leftmargin}{1.5em}
     \setlength{\labelwidth}{1em}
     \setlength{\labelsep}{0.5em} } }
\newcommand{\squishend}{
  \end{list}  }

\newenvironment{packed_enum}{
\begin{enumerate}
  \setlength{\itemsep}{1pt}
  \setlength{\parskip}{0pt}
  \setlength{\parsep}{0pt}
}{\end{enumerate}}

\newenvironment{packed_desc}{
\begin{description}
  \setlength{\itemsep}{1pt}
  \setlength{\parskip}{0pt}
  \setlength{\parsep}{0pt}
}{\end{description}}

%
% paper title
% Titles are generally capitalized except for words such as a, an, and, as,
% at, but, by, for, in, nor, of, on, or, the, to and up, which are usually
% not capitalized unless they are the first or last word of the title.
% Linebreaks \\ can be used within to get better formatting as desired.
% Do not put math or special symbols in the title.
\title{BrainReader: Effective Visualization\\of fMRI-based Movie Reconstruction}
%
%
% author names and IEEE memberships
% note positions of commas and nonbreaking spaces ( ~ ) LaTeX will not break
% a structure at a ~ so this keeps an author's name from being broken across
% two lines.
% use \thanks{} to gain access to the first footnote area
% a separate \thanks must be used for each paragraph as LaTeX2e's \thanks
% was not built to handle multiple paragraphs
%

\author{Natalia Bilenko, Valkyrie Savage% <-this % stops a space
\thanks{Natalia and Valkyrie are joint first-authors on this work.  They can be reached via email at \emph{nbilenko@berkeley.edu} and \emph{valkyrie@eecs.berkeley.edu}.}% <-this % stops a space
\thanks{Manuscript submitted December 18th, 2014.}}



% The paper headers
\markboth{Computational Photography Final Project, CS 294-84, December 2014}%
{Bilenko and Savage: BrainReader}



% If you want to put a publisher's ID mark on the page you can do it like
% this:
%\IEEEpubid{0000--0000/00\$00.00~\copyright~2014 IEEE}
% Remember, if you use this you must call \IEEEpubidadjcol in the second
% column for its text to clear the IEEEpubid mark.



% use for special paper notices
%\IEEEspecialpapernotice{(Invited Paper)}




% make the title area
\maketitle

% As a general rule, do not put math, special symbols or citations
% in the abstract or keywords.
\begin{abstract}
Previous work in decoding visual experiences based on fMRI activity has been successful in reconstructing images and movies that participants viewed inside an MRI scanner. Reconstruction is done by fitting a forward model that predicts fMRI activity across the brain in response to a set of movies. The model represents brain activity as a linearized function of visual information features that capture the structure of the movies (spatiotemporal Gabor wavelet filters). The forward model is then inverted and used to decode what the subject saw based on their brain responses to a testing set of movies. Decoding is performed by fitting a maximum a posteriori function to a large library of previously unseen movie clips. The top 100 decoded movie clips are then averaged or stitched together to produce a visualization of the decoding. Though the decoding is quite precise when measured quantitatively, these visualizations do not fully reflect its accuracy. We make the visualization more coherent by combining the decoded clips in several improved ways. First, we demonstrate the change in quality gained using weighted averaging.  Then, we use HOG features to select a subset clips similar to the ground truth clip and SIFT flow to find an optimal path in time. Third, we use appearance morphing to visually align the path-arranged clips. Finally, we share the decoded movies resulting from the same stimuli across different participants in the experiment.
\end{abstract}

% Note that keywords are not normally used for peerreview papers.
\begin{IEEEkeywords}
fMRI, decoding, visualization, computational videography.
\end{IEEEkeywords}


\IEEEpeerreviewmaketitle

%!TEX root = brainreader.tex

\section{Introduction}
\IEEEPARstart{F}{unctional} magnetic resonance imaging, or fMRI, data can shed light on what individuals are looking at.  Specific areas in the brain are known to react strongly to particular line orientations and locations.  Using activation data from these centers, such fMRI data can ``reconstruct'' what an individual is seeing.

The Gallant lab from UC Berkeley has worked on this problem previously, demonstrating \valkyrie{citation needed} that they can perform this reconstruction by averaging images from a training set.  The top 100 images whose recorded fMRI profiles match most closely with the recorded data are simply stacked on top of each other to create output videos.

However, this type of visualization is very ``messy'': while the quantified fMRI matches are quite strong, the visual output video stacks are misleadingly inaccurate.  Using computational techniques, we can improve the quality of these output videos (Figure \ref{fig:avg})


\begin{figure}[t]
\centering
    \includegraphics[width=1.0\columnwidth]{figures/average.png}
\caption{a) The first generation of visualization for BrainReader clips: the top 100 guesses are simply averaged and overlaid to create an output image at each frame.  This belies the accuracy of the technique, which is quite high. b) Our improved visualization \valkyrie{obviously need image here.}.}
\label{fig:avg}
\end{figure}


Instead of a simple averaging process for turning guesses into an output video, we used several more sophisticated approaches.  First, we simply performed weighted averaging, using clip ranking as our weights.  This already constituted an improvement over na\"{i}ve weighting (Figure \ref{fig:weighted}).

\begin{figure}[t]
\centering
    \includegraphics[width=1.0\columnwidth]{figures/average.png}
\caption{Using a weighted average improves over the na\"{i}ve average seen in Figure \ref{fig:avg}a. \valkyrie{obviously we need this image as well.}}
\label{fig:weighted}
\end{figure}


After using weighted averaging, we turned to even more sophisticated techniques: HOG feature alignment and SIFT flow.  These give us additional information on the edges makeup and semantic scene composition of both the original clip and the top 100 guesses as predicted by the original fMRI-based system.

%!TEX root = brainreader.tex

\section{Methods}

Our processing pipeline takes as input two pieces of data for each second:

\begin{itemize}
\item The original clip stimulus presented to the subject
\item The top 100 guesses (based on fMRI data) and their rankings
\item The log-likelihood (LLH) of each guess clip based on the fMRI data
\end{itemize}

We first perform HOG feature extraction on both pieces of data and reject guesses whose HOG data does not match well with the original clip.  We then extract SIFT features from the beginning and ending of each of the top guesses and calculate SIFT flow between them.  Using cost back-propagation, we find the lowest-cost path through the remaining clips.  \valkyrie{Finally, we perform morphing between these clips using SIFT keypoints and output the final clip compilations.}

Clips (both original and guessed) are 1 second in length, and have 15 frames per second.  We do not have data on whether there are scene breaks within a 1 second clip.  

\subsection{Pruning - HOG Features}

Histogram of Oriented Gradients (HOG) features roughly indicate edges in an image as well as the orientation of those edges.  We use HOG features to ensure good \emph{spatial} alignment of guess footage with the presented clip.

We calculate the HOG features of each frame of each clip and perform an SSD with the ground truth clip's HOG features at that time step.  This process is physically based in the fMRI data, as the visual processing centers of the brain react in specific ways to edges presented in particular orientations and in particular locations across the visual field.  Thus we see this ``pruning'' step as non-essential, as we would expect that the fMRI data and subsequent ranking step (performed prior to our getting the data) is already based on these features.   \valkyrie{We should look at how well the HOG features actually correlate to rankings.  I assume we basically pick the top guesses, but I don't know.}

After finding the SSD of the HOG features through each clip when compared to the original source clip, we throw away \valkyrie{what exactly do we throw away?}.

\subsection{Consistency - SIFT Features}

Scale-Invariant Feature Transform (SIFT) features, often used in image recognition tasks, can give higher-level information about the contents of a scene.  We want to minimize the key point flow (i.e., scene composition) between the last frame of one clip and the first frame of the next clip.  SIFT keypoints have been used for nearest-neighbor database searches (e.g., in the SIFT flow paper), and can successfully extract, for example, a street image to match a street image, even when the \emph{optical} flow between the two street images is large.  We use this to keep a thematically consistent scene across timesteps.

We use SIFT flow to calculate costs for transitioning between one clip and the next.  We simply calculate the SIFT flow between the last frame of one clip and the first frame of all potential next clips.  We then use cost back-propagation, i.e., dynamic programming, to find the lowest semantic cost path through all the clips remaining after HOG pruning.

\subsection{Visuals - Image morphing}

To make our output visuals more consistent, we introduce an aspect of warping between the final selected clips.  This warping is intended to preserve object motion and position across clip boundaries: for example, a person walking in one clip should not jump to a new location in the next clip, and she should not drastically change in appearance at the clip boundary.  In addition, we are also confined to the space of clips already selected, and we do not add any new or extended clips to improve overlap information.

This step also uses SIFT flow.  We c

%!TEX root = brainreader.tex

\section{Related Work}

Our work relates to both basic image processing and video processing.

%!TEX root = brainreader.tex

\section{Discussion}

As our understanding of the brain becomes more and more precise, this type of visualization will surely get better.  This iteration of the BrainReader project lacks information about colors and semantics: addition of both would certainly lead to even more accurate results.

Additionally, some results in our dataset are certainly more convincing than others.  Data for which there are many nearly-precise matchings in the database (for example, of human faces) lead to much nicer visualizations, while more unusual data (for example, nudibranchs and other sea creatures) must have their edges "re-constituted" by many near matches overlaid.  This is a limitation of the dataset itself; while our alignment and visualization techniques can help, it is challenging to overcome holes in the data.  \valkyrie{we should have a figure here of some failed things from our pipeline.}

In future work, it would be beneficial to quantitatively compare our visualization techniques with the decoded fMRI data: for example by measuring distances between presented edges and edges in the final visualization, or by presenting the visualization to subjects to compare the brain's reaction to it with the brain's reaction to the original stimulus.

%!TEX root = brainreader.tex

\section{Conclusion}

We have compared a variety of processing techniques for creating effective visualizations of fMRI data.  Our visualization processing constitutes a huge improvement over previous work, and better indicates the precision of the fMRI technique.




% if have a single appendix:
%\appendix[Proof of the Zonklar Equations]
% or
%\appendix  % for no appendix heading
% do not use \section anymore after \appendix, only \section*
% is possibly needed

% use appendices with more than one appendix
% then use \section to start each appendix
% you must declare a \section before using any
% \subsection or using \label (\appendices by itself
% starts a section numbered zero.)
%


%\appendices
%\section{Proof of the First Zonklar Equation}
%Appendix one text goes here.

% you can choose not to have a title for an appendix
% if you want by leaving the argument blank
%\section{}
%Appendix two text goes here.


% use section* for acknowledgment
\section*{Acknowledgments}
The authors would like to thank Alyosha, Shiry, and Shubham for being generally awesome, teaching neat material, and giving suggestions on the project.  What a fun semester!\\
We would also like to thank the Gallant lab, in particular Shinji Nishimoto, for providing us with the data necessary for this project.


% Can use something like this to put references on a page
% by themselves when using endfloat and the captionsoff option.
\ifCLASSOPTIONcaptionsoff
  \newpage
\fi


% trigger a \newpage just before the given reference
% number - used to balance the columns on the last page
% adjust value as needed - may need to be readjusted if
% the document is modified later
%\IEEEtriggeratref{8}
% The "triggered" command can be changed if desired:
%\IEEEtriggercmd{\enlargethispage{-5in}}

% references section

% can use a bibliography generated by BibTeX as a .bbl file
% BibTeX documentation can be easily obtained at:
% http://www.ctan.org/tex-archive/biblio/bibtex/contrib/doc/
% The IEEEtran BibTeX style support page is at:
% http://www.michaelshell.org/tex/ieeetran/bibtex/
%\bibliographystyle{IEEEtran}
% argument is your BibTeX string definitions and bibliography database(s)
%\bibliography{IEEEabrv,../bib/paper}
%
% <OR> manually copy in the resultant .bbl file
% set second argument of \begin to the number of references
% (used to reserve space for the reference number labels box)
\bibliographystyle{IEEEtran}
\bibliography{references}


% biography section
% 
% If you have an EPS/PDF photo (graphicx package needed) extra braces are
% needed around the contents of the optional argument to biography to prevent
% the LaTeX parser from getting confused when it sees the complicated
% \includegraphics command within an optional argument. (You could create
% your own custom macro containing the \includegraphics command to make things
% simpler here.)
%\begin{IEEEbiography}[{\includegraphics[width=1in,height=1.25in,clip,keepaspectratio]{mshell}}]{Michael Shell}
% or if you just want to reserve a space for a photo:

\begin{IEEEbiography}[{\includegraphics[width=1in,height=1.25in,clip,keepaspectratio]{figures/natalia.JPG}}]{Natalia Bilenko}
Natalia scans brains with large magnets.  She is a grad student in Jack Gallant's lab at the Helen Wills Neuroscience Institute.
\end{IEEEbiography}

\begin{IEEEbiography}[{\includegraphics[width=1in,height=1.25in,clip,keepaspectratio]{figures/valkyrie.png}}]{Valkyrie Savage}
Valkyrie is a grad student whose desk is covered in 3D printed stuff.  She works for Bjoern Hartmann in the Berkeley Institute of Design.
\end{IEEEbiography}


% insert where needed to balance the two columns on the last page with
% biographies
%\newpage

% You can push biographies down or up by placing
% a \vfill before or after them. The appropriate
% use of \vfill depends on what kind of text is
% on the last page and whether or not the columns
% are being equalized.

%\vfill

% Can be used to pull up biographies so that the bottom of the last one
% is flush with the other column.
%\enlargethispage{-5in}



% that's all folks
\end{document}


